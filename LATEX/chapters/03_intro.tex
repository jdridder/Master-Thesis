\chapter{Introduction}\label{intro} 

In the chemical process industry, system models are a fundamental tool for design and operation. Due to its
techno-scientific nature, chemical engineering is about modelling physical and chemical phenomena in an
accurate enough manner to realize a save and profitable conversion process. These models vary in complexity and 
prediction potency from single equations such as the Bernoulli equation to detailed partial differentail equation systems
such as multi-phase Navier-Stokes. Physico-chemical models are at the core of optimal design or optimal control
of most unit operations.
\newline
As these \emph{first principle models} rely on intrinsic parameters such as activation energies, friction coefficients, etc.
that are usually fitted by experiments, they exhibit some kind of prediction uncertainty. This uncertainty is 
heavily tied to the measurement uncertainty of the underlying experiments and the non-linearity in the propagation by the model. 
For example parametric uncertainties of $ \pm 50~\%$ can occur in multi-component reaction systems \cite{empty000}.
As most phenomena in chemical engineering exhibit strong non-lineatities, such as reaction kinetics or activity coefficients,
uncertainty quantification represents a great challenge. 
\newline
\emph{Model Predictive Control} (MPC) is a potent optimal control algorithm that allows the optimal operation of various systems. 
The algorithm optimizes a custom loss or reward function under a set of constraints and the dynamics of the system model. The optimal inputs
of the system are determined at every time instant as the solution of the optimization. MPC is able do deal with uncertain parameter
scenarios in the form of \emph{multi-stage} MPC, where all possible scenarios are optimized over simultaneously. This leads
to tremendous computational effort especcialy for detailed system models. The computational effort growths multiplicatory with the 
amount of parameters and exponentially with the robust horizon \cite{empty000}. These optimization tasks are impossible to solve
in real time with current hardware.
\newline
\emph{Neural Networks} (NNs) are able to approximate any continuous function that maps from a finite dimensional space to
another up to an arbitrary small error \cite{empty000}. Thus, NNs are a key component of machine learning advancements of the 2010s enabling
data approximation of almost any kind such as autoregressive language generation \cite{vaswani2023}, protein folding \cite{empty000} or
reduction of partial differential equations (PDEs). Their universal approximation ability, differentiatability and fast evaluation makes NNs greatly suitable to reduce first principle models in
the context of optimal control.
\newline
For NNs to be applied successfully and savely in process engineering, another measure is of great importance – The alignment of 
predictions with physical conservation laws such as mass or energy. By definition, NN predictions seek to minimize a loss function
to the ground truth data. The aligment of the predictions with real balance equations is only achieved until test accuracy. Deviations
accumulate especcialy in propagation scenarios where predictions are propagated through different networks or through a time horizon
in autoregressive manner. This leads to bad predictions or even unstable behavior in late propagation stages and makes MPC
implementations challenging.
\newline
This work investigates the usage of physics-consistent (PC) NNs in the context of robust model predictive control under
parametric uncertainty. The NNs in focus shall significantly reduce the computational effort due to different parameter scenarios as well as 
guarantee physical correctness of the predictions over the whole prediction horizon. To account for uncertainties, quantile
regression is utilized which captures a prediction confidence intervall. PC is enforced by implementing a recent
type of activation function that elegantly maps a prediction into the allowed output space without altering the inference time \cite{chen2024}.
\newline
The developed surrogate models are later evaluated regarding their simulation accuracy using a fictional plug flow tubular reactor (PFTR) model of the ethene oxide synthesis.

