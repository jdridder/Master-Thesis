\chapter{Model Predictive Control under Uncertainty}


Before diving into the application of a special model predictive (MPC) algorithm. Uncertain
optimization is treated more generally. From economics, over self-driving cars to chemical plants,
optimization problems are formulated in terms of uncertain variables in countless variations.
These uncertainties arise in three different types – fluctuating disturbances such as wind or weather conditions,
parametric offsets such as wrongly determined friction coefficients or model mismatches like unseen side reactions.
As the straight maximization or minimzation of an objective often puts a problem to its boundaries,
unforeseen disturbances may lead to \emph{constraint violation} or even \emph{instability} \cite{houska2011}.
Therefore, \emph{robust} optimization formulations are needed, that are proof against mentioned challenges as
well as retain the \emph{performance} with respect to the objective function.
\newline

\section{Robust Optimization}

Considering the following optimization task (\ref{theroy:def: semi-infinite program}).
\begin{definition} Semi-infinite program with the uncertainty $\vw$.
    \label{theroy:def: semi-infinite program}
    \[
    \begin{gathered}
        \underset{\vu}{\min} \; \underset{\vw}{\max} \quad F_0(\vu, \vw) \\
        \text{s.t.} \quad F_i(\vu, \vw) \leq 0 \quad \forall \vw \in \mathbb{W}
    \end{gathered}
    \]
\end{definition}

Here, the optimization of the objective is formulated in a bilevel form. The scalar objective function $F_0$ is to 
be minimized w.r.t the decision variables $\vu$ for the worst case scenario w.r.t the disturbance
vector $\vw$. The inequality constraints $F_i$ however, should be satisfied for all possible disturbances, that
lie in a defined set $\mathbb{W}$. If this set is not finite, the number of possible 
constraints to be satisifed is infintely big. Thus, this formulation is named as \emph{semi-infinite-program} \cite{houska2011}.
To reduce the number of constraints, the worst-case estimate can also be applied here (\ref{theroy:def: bi-level program}). 

\begin{definition} Bi-level formulation of the semi-infinite program
    \label{theroy:def: bi-level program}
    \[
    \begin{gathered}
        \underset{\vu}{\min} \; \underset{\vw}{\max} \quad F_0(\vu, \vw) \\
        \text{s.t.} \quad \varphi_i(\vu) \coloneq \underset{\vw}{\max} \quad F_i(\vu, \vw) \leq 0
        \quad \vw \in \mathbb{W}
    \end{gathered}
    \]
\end{definition}

Now, the inequality constraints are also formulated in a bi-level manner. The maximum of $F_i$ is 
constant in $\vw$ and only a function of the decision variables $\vu$. The function $\varphi_i$
exactly describes the wost-case for $F_i$ parametrized by $\vu$. In fact, if certain assumptions hold
regarding the uncertainty set $\mathbb{W}$ and the concavity of the maximization,
$\varphi_i$ can be exactly computed \cite{empty000}. This would simplify the problem dramatically.
In reality however, an analytical solution for $\varphi_i$ is not possible to compute.
But with the universal approximation capability of neural networks (NNs) in mind \cite{hornik1990}, 
a data-driven approach is promising, that delivers an approximated function $\tilde{\varphi}_i(\vu)$.

\section{Robust Model Predictive Control}

(A short intro to MPC)
A model predictive control algorithm solves a control optimization task at each time instant.
The dynamic optimization anticipates the future behavior of the controlled system through
a mathematical model and determines the control inputs as the solution. Only part of the 
solution is applied as a control input until the next sampling instant. To yield correct inputs,
the prediction quality of the underlying model is of crucial importance.
\newline
\newline

Coming from this general perspective on robust optimization, \emph{robust MPC} specializes
this concept to a dynamic closed-loop control environment. First, the continuous-time
formulation is considered.

\begin{definition} Continuous-time formulation of an MPC task under uncertainty.
    \label{theory:def: continuous mpc}
    \[
        \begin{gathered}
            \underset{\vu}{\min} \; \underset{\vw}{\max} \quad  J(t_0, t_\text{end}, \vx, \vu, \vw) \\
            \begin{aligned}
                \text{s.t.} \quad &\dot{\vx} = f(t, \vx, \vu, \vw) \\
                & F_i(t, \vx, \vu, \vw) \leq 0 \quad \forall t \in [t_0, t_\text{end}], \; \forall \vw \in \mathbb{W} \\
                & \vx_0 = \vx_\text{init}
            \end{aligned}
        \end{gathered}
    \]
\end{definition}

The only difference to the general problem (\ref{theroy:def: semi-infinite program}) is that the states, denoted with $\vx$,
evolve along some system dynamics that are described by $f$, that also underlie to the uncertainty $\vw$.
This formulation is now infinite in time and in the uncertain parameters. To make computation feasible,
the problem is discretized in the time dimension eiher using single shooting, multiple shooting or orthogonal collocation
on finite elements \cite{}. One possibility to roughly discretize the uncertain parameter set $\mathbb{W}$ is
done by \emph{multi-stage-MPC} \cite{}. Which is intuitive, but leads to tremendous computational effort, especcialy 
with growing dimensionality of the uncertainty space.

\begin{definition} Time-discretized multi-stage MPC
    \label{theory:def: multi-stage-mpc}
    \[
        \begin{gathered}
            \underset{\vu_k, \vx_k}{\min} \; \underset{\vw_k}{\max}  \quad  J(N_\text{p}, \vx_k, \vu_k, \vw_k)
        \end{gathered}
    \]
\end{definition}



\section{Uncertainty}
Model uncertainties are a main challenge in MPC. These uncertainties can occur in the form of measurement noise, unknown parameters such as 
activation energies or fluctuating disturbances such as wind from different directions. In the context of chemical engineering, the
focus lies on parametric uncertainty of reaction kinetic parameters. Eventhough, the underlying methodlogy can be generalized to any kind of 
model uncertainty.
\newline
The afforementioned uncertainties can be grouped into two types (eq.~\ref{eq: uncertainties}).
\begin{equation}
    \label{eq: uncertainties}
    \begin{aligned}
        \bm{x}_{k+1} &= f(\bm{x}_k, \bm{u}_k) + \bm{w}_k, \quad \bm{w}_k \in \mathbb{W}
        \\
        \bm{x}_{k+1} &= f(\bm{x}_k, \bm{u}_k, \bm{d}_k), \quad \bm{d}_k \in \mathbb{D}
    \end{aligned}
\end{equation}
The first one is additive disturbance and much easier to handle than the second type that propagates the disturbances non-linearly
through the system \cite{empty000}. The second type is more general and most parametric uncertainties occur according to this type, which is why
it is investigated in this work. 

Conservartiveess vs Performance

\section{Multi-Stage MPC}








\chapter{Surrogate Modeling}

\section{Neural Networks}

Feed forward neural networks (NNs) provide a stunning capability of approximating any non-linear function from any finite dimensional
space to another. According to the \emph{Universal approximation theorem}, any finite dimensional function can be reproduced if the number of hidden
layers and respective neurons is high enough, up to an arbitrarily small error \cite{hornik1990}. Thus, NNs play a major role in
recent developments in machine learning. The breakthrough of large language models such as ChatGPT heavily rely on neural
networks [\cite{vaswani2023}]. NNs are applied not only in natural lanugage processing but in the chemical process industry.
Use cases range from thermodynamic property prediction over unit operation design to corporate level decision making [\cite{mcbride2019}].
Their ability to approximate any non-linear function while beeing differentiable, makes NNs especcialy suitable for dynamic optimization
such as MPC. In contrast to detailed first principle models, NNs have the advantage of making a prediction in a fraction of the inference time
of the first principle model. As MPC operates in real time parallel to the system to be controlled, the computation duration is a key measure
for a succesfull MPC implementation. Thus, surrogate models in form of NNs are a crucial tool for dynamic optimization and heavily investigated
in current research. May a NN be given as a alternating series of function evaluations (eq.~\ref{eq: neural network}).

\begin{equation}
    \label{eq: neural network}
    \begin{aligned}
        \mathcal{NN}(\bm{x, \bm{\theta}}) &= g_{l+1} \circ h_{l+1} \circ ... \circ g_1 \circ h_1(\bm{x})
        \\
        h(\bm{a}_{l-1}) &= \bm{\theta}_{Wl}\bm{a}_{l-1} + \bm{\theta}_{bl}
        \\
        \bm{a}_{l} &= g_l(h(\bm{a}_{l-1}))
    \end{aligned}
\end{equation}

In fact, a NN is a series of function combinations denoted by the $\circ$-symbol alternating between a linear function $h(\bm{a})$
and a non-linear activation function $g$. The linear transformation matrices $\bm{\theta}_{Wl}$ and offsets $\bm{\theta}_{bl}$,
are called weights and biases respectively and are the trainable parameters of a NN. Given a set of training data $\mathbb{D}_\text{train}$,
the optimal parameters for $\bm{\theta}$ are found by solving an optimization task that involves the minimzation of a prediction loss function (eq.~\ref{eq: loss function}).

\begin{equation}
    \label{eq: loss function}
    \begin{aligned}
        \mathbb{X}_\text{train} &\coloneq \{ \bm{\xi}^{(0)}, \hdots, \bm{\xi}^{(m)} \}
        \\
        \mathbb{Y}_\text{train} &\coloneq \{ \bm{y}^{(0)}, \hdots, \bm{y}^{(m)} \} 
        \\
        \mathcal{L}(\bm{\theta}, \mathbb{X}_\text{train}, \mathbb{Y}_\text{train}) &=
        \frac{1}{m} \sum_{i=0}^{m-1} \Big| \Big| \bm{y}^{(i)} - \bm{\tilde{y}}(\bm{\theta}, \bm{\xi}^{(i)}) \Big | \Big|_2^2
    \end{aligned}
\end{equation}

In \emph{supervised learning}, the training data is separated in \emph{features} $\mathbb{X}_\text{train}$, the inputs of the NN,
and \emph{labels} $\mathbb{Y}_\text{train}$, the outputs of the NN. During training, the parameters $\bm{\theta}$ are determined such that
the NN approximates the data in best possible manner.
In this example the loss function $\mathcal{L}$ portrays the mean squarred error (MSE) that is used in most regression tasks. 
The difference between the model prediction, denoted as $\bm{\tilde{y}}$, and the true data point $\bm{y}^{(i)}$ is penalized quadratically.
The kind of loss function heavily impacts the characteristics of the NN model prediction.

Explain batching
Batching allows to pass multiple data points forward
through the NN.

