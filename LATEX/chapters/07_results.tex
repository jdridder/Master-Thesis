\chapter{Results on Surrogate Model Performance}
Please choose a more expressive title for this chapter. If it makes sense for your work you can also have multiple chapters presenting your results.

\section{Training Effectivity}

% \begin{figure}
%     \centering
%     \includegraphics[width=1\textwidth]{Figures/Results/001_val_loss.pdf}
%     \caption[Validation loss of vanilla NARX vs physics-constrained NARX]{The validation loss calculated as MSE $\mathcal{L}_\mathrm{MSE, val}$
%     during the training of the vanilla NARX (dashed line) and the physics-constrained NARX (solid line). The three data rows each indicate
%     \emph{one hidden layer} with $64, 16$ and $4$ hidden neurons respectively. The physics-constrained NARX converges visibly faster
%     than the vanilla NARX.}
%     \label{intro:fig:01_data}
% \end{figure}


\section{Physics Consistency}

To analyze the physics consistency of the different NARX frameworks an open-loop simulation is performed.
It propagates the state trajectories for the nominal case, the upper and the lower quantiles forward
in time independently (def.~\ref{sur:def: general rhs for uncertainty quantification}). This is done for 
the three framework variations \emph{vanilla}, \emph{naive\_pc} and \emph{pc}. The main 
quality metric is the balance violation of the conservation of atoms due to stoichiometry (th.~\ref{case:cor: conservation of atoms}).
When it is equal to the zero vector, it is perfectly fulfilled.
The residual of the linear balance constraint is calculated (eq.~\ref{results:eq: residual of balance equation})
as the $l_2$-norm
and plotted as a function of time (fig.~\ref{results:fig: pc_violation_vanilla}).
\begin{equation}
    \label{results:eq: residual of balance equation}
    ||\bm{b}||_2 = ||\bm{B} \bm{\Delta \xi}||_2
\end{equation}

\begin{figure}
    \centering
    \includegraphics[width=1\textwidth]{Figures/Results/pc_violation_vanilla.pdf}
    \caption[Physics violation of the vanilla NARX]
    {Balance equation violation as a function of time in a \SI{480}{\second} forward simulation
    of the vanilla NARX framework. The nominal trajectory $\mathbb{E}$ (green) displays slightly better alignment
    with the balance than the models conditioned by quantiles $Q_{0.9}$ and $Q_{0.1}$.}
    \label{results:fig: pc_violation_vanilla}
\end{figure}

To indicate maximum machine precision, horizontal dashed lines are drawn for \SI{64}{\bit} (gray)
and \SI{32}{\bit} (black). The balance residual norm $||\vb||_2$ is shown for the nominal NARX (green), trained
with a standard MSE, and the conditioned NARX models, that are trained with a weighted MSE.
Here, yellow colors the NARX conditioned by the $0.9$-quantile and purple the NARX conditioned by the $0.1$-quantile.
After beeing initialized with an initial state, that obeys the balance up to an error of \SI{1e-14}{}. 
All NARX models abruptly increase the error to \SI{1e-3} and fluctuate in this region until the simulation
horizon is reached. When the residual is averaged over the simulation horizon using the $\langle(\cdot)\rangle_t$
notation, the nominal NARX model produces a slightly smaller error of \SI{6.39e-4}{}.
In contrast, the error of the quantile-conditioned NARX models is rougly twice as big with values
of \SI{1.12e-3} and \SI{1.42e-3} for the $0.9$ and $0.1$ quantile respectively (fig.~\ref{results:fig: pc_violation_vanilla}).
It can be noted that neither an instability nor a divergence occurs
that would lead to a drastic increase in the residual error.
Additionally, the error does not add up over time.

\begin{figure}
    \centering
    \includegraphics[width=1\textwidth]{Figures/Results/pc_violation_naive_pc.pdf}
    \caption[Physics violation of the naive physics-constrained NARX]
    {Balance equation violation as a function of time in a \SI{480}{\second} forward simulation
    of the vanilla NARX framework. The nominal trajectory $\mathbb{E}$ (green) displays slightly better alignment
    with the balance than the models conditioned by quantiles $Q_{0.9}$ and $Q_{0.1}$.}
    \label{results:fig: pc_violation_naive}
\end{figure}



\begin{figure}
    \centering
    \includegraphics[width=1\textwidth]{Figures/Results/pc_violation_pc}
    \caption[Physics violation of the physics-constrained NARX]
    {Balance equation violation as a function of time in a \SI{480}{\second} forward simulation
    of the vanilla NARX framework. The nominal trajectory $\mathbb{E}$ (green) displays slightly better alignment
    with the balance than the models conditioned by quantiles $Q_{0.9}$ and $Q_{0.1}$.}
    \label{results:fig: pc_violation_pc}
\end{figure}
