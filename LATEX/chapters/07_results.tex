\chapter{Results on Surrogate Model Performance}
Please choose a more expressive title for this chapter. If it makes sense for your work you can also have multiple chapters presenting your results.

\section{Training Effectivity}

% \begin{figure}
%     \centering
%     \includegraphics[width=1\textwidth]{Figures/Results/001_val_loss.pdf}
%     \caption[Validation loss of vanilla NARX vs physics-constrained NARX]{The validation loss calculated as MSE $\mathcal{L}_\mathrm{MSE, val}$
%     during the training of the vanilla NARX (dashed line) and the physics-constrained NARX (solid line). The three data rows each indicate
%     \emph{one hidden layer} with $64, 16$ and $4$ hidden neurons respectively. The physics-constrained NARX converges visibly faster
%     than the vanilla NARX.}
%     \label{intro:fig:01_data}
% \end{figure}


\section{Physics Consistency}

To analyze the physics consistency of the different NARX frameworks an open-loop simulation is performed.
It propagates the state trajectories for the nominal case, the upper and the lower quantiles forward
in time independently (def.~\ref{sur:def:general rhs for uncertainty quantification}). This is done for 
the three framework variations \emph{vanilla}, \emph{naive\_pc} and \emph{pc}. The main 
quality metric is the balance violation of the conservation of atoms due to stoichiometry (th.~\ref{case:cor: conservation of atoms}).
When it is equal to the zero vector, it is perfectly fulfilled.
The residual of the linear balance constraint is calculated (eq.~\ref{results:eq:residual of balance equation})
as the $l_2$-norm
and plotted as a function of time (fig.~\ref{results:fig:pc_violation_vanilla}).
\begin{equation}
    \label{results:eq:residual of balance equation}
    ||\bm{b}||_2 = ||\bm{B} \bm{\Delta \xi}||_2
\end{equation}

\begin{figure}
    \centering
    \includegraphics[width=1\textwidth]{Figures/Results/pc_violation_vanilla.pdf}
    \caption[Physics violation of the vanilla NARX]
    {Balance equation violation as a function of time in a \SI{480}{\second} forward simulation
    of the vanilla NARX framework. The nominal trajectory $\mathbb{E}$ (green) displays slightly better alignment
    with the balance than the models conditioned by quantiles $Q_{0.9}$ and $Q_{0.1}$.}
    \label{results:fig:pc_violation_vanilla}
\end{figure}

To indicate maximum machine precision, horizontal dashed lines are drawn for \SI{64}{\bit} (gray)
and \SI{32}{\bit} (black). The balance residual norm $||\vb||_2$ is shown for the nominal NARX (green), trained
with a standard MSE, and the conditioned NARX models, that are trained with a weighted MSE.
Here, yellow colors the NARX conditioned by the $0.9$-quantile and purple the NARX conditioned by the $0.1$-quantile.
After beeing initialized with an initial state, that obeys the balance up to an error of \SI{1e-14}{}. 
All NARX models abruptly increase the error to \SI{1e-3} and fluctuate in this region until the simulation
horizon is reached. When the residual is averaged over the simulation horizon using the $\langle(\cdot)\rangle_t$
notation, the nominal NARX model produces a slightly smaller error of \SI{6.39e-4}{}.
In contrast, the error of the quantile-conditioned NARX models is roughly twice as big with values
of \SI{1.12e-3} and \SI{1.42e-3} for the $0.9$ and $0.1$ quantile respectively (fig.~\ref{results:fig:pc_violation_vanilla}).
It can be noticed that neither an instability nor a divergence occurs
that would lead to a drastic increase in the residual error.
Additionally, the error does not add up over time.

\begin{figure}
    \centering
    \includegraphics[width=1\textwidth]{Figures/Results/pc_violation_naive_pc.pdf}
    \caption[Physics violation of the naive physics-constrained NARX]
    {Balance equation violation as a function of time in a \SI{480}{\second} forward simulation
    of the vanilla NARX framework. The nominal trajectory $\mathbb{E}$ (green) displays slightly better alignment
    with the balance than the models conditioned by quantiles $Q_{0.9}$ and $Q_{0.1}$.}
    \label{results:fig:pc_violation_naive}
\end{figure}

The same is done with the \emph{naive\_pc} NARX models that are not trained with the physics constrained
activation but simply apply it to their output (fig.~\ref{results:fig:pc_violation_naive}). The 
balance error during the simulation shows a completely different picture. All models achieve
vanishing balance errors at \SI{64}{bit} precision of \SI{3.4e-16}{}. These appear consistent over the complete
simulation horizon. The naively physics constrained models exhibit even smaller balance errors than the initial state itself, which stems
from a first principle simulation, due to fewer rounding errors. There is no noticable difference
between the nominal predictions $\mathbb{E}$ (green) and the conditioned quantile predictions $Q_{0.9}$ (yellow) and $Q_{0.1}$ (purple)
in contrast to the vanilla versions. Additionally, it can be interpreted as a proof of concept
for the physics constrained activation $g_\text{pc}$. It enforces linear equality constraints up to machine precision, even
if it is not used during training.


\begin{figure}
    \centering
    \includegraphics[width=1\textwidth]{Figures/Results/pc_violation_pc}
    \caption[Physics violation of the physics-constrained NARX]
    {Balance equation violation as a function
    
    of time in a \SI{480}{\second} forward simulation
    of the vanilla NARX framework. The nominal trajectory $\mathbb{E}$ (green) displays slightly better alignment
    with the balance than the models conditioned by quantiles $Q_{0.9}$ and $Q_{0.1}$.}
    \label{results:fig:pc_violation_pc}
\end{figure}

Finally, the balance error is plotted for the \emph{pc}-NARX, that uses the physics constrained
activation during training and simulation (fig.~\ref{results:fig:pc_violation_pc}). While it significantly improves
training efficiency, it also enforces linear equality constraints in an open-loop simulation up to \SI{32}{\bit} precision.
Similarly to the \emph{naive\_pc}, all three NARX models achieve the same contraint residual error
of \SI{1.7e-7}{}. The difference between the physics constrained nominal trajectory (green) and the quantile conditioned (yellow, purple)
trajectories is non-existent. It is worth noticing, that the minimum possible contraint residual differs 
between the \emph{naive\_pc} and the \emph{pc}-NARX. The \emph{naive} version is not trained with the 
physics consistent activation $g_\text{pc}$. It is applied \emph{a posteriori} and implemented using
a \texttt{casadi}-function object. The \emph{pc} version owns a direct implementation of $g_\text{pc}$
as a \texttt{torch.nn.Module}. Eventhough it is converted into a \texttt{casadi} expression to be
simulated in \texttt{do\_mpc} afterwards, it still keeps the \SI{32}{\bit} precision of the \texttt{torch.float32}-object.
Because the \texttt{casadi}-function computes in the \SI{64}{\bit} decimal space, \emph{naive} and \emph{pc}
differ in maximum machine precision. Since the residual norm shrinks to insignificance
with values at \SI{1e-7} or \SI{1e-16} in both cases, \SI{32}{\bit} precision is not considered
disadvantageous.


\section{Simulation Capability}

To investigate, whether the NARX framework is able to track the distribution of state trajectories, 
the simulation results for the expected value and quantile NARX models are illustrated (fig.~\ref{results:fig:vanilla_temperature_sim}).
\begin{figure}
    \centering
    \includegraphics[width=1\textwidth]{Figures/Results/test_with_vanilla_-1.pdf}
    \caption[Forward simulation of the vanilla NARX framework for the temperature.]
    {Forward simulation of the vanilla NARX framework without physics constrained activation for 
    \SI{480}{\second}. The scaled temperature $T'$ is shown for two measurement positions $z=0.25L$ and
    $z=L$. $100$ first principle system trajectories are displayed as test data (light blue) for MC sampled parameters.}
    \label{results:fig:vanilla_temperature_sim}
\end{figure}

First, a look at the temperature is taken, which undergoes quantile regression and state propagation via
the \emph{vanilla} framework (dashed line). The measurement positions for display are choosen to be the reactor outlet $z=L$ and 
the first quarter $z=0.25L$, where a hotspot is likely to form. While yellow, green and purple
indicate the $0.9$-temperature quantile, expected value and $0.1$-temperature-quantile, the ground truth
test trajectories are marked in light blue. These originate from first principle model simulations
with MC sampled parameter combinations along the normal distributions. By their transparent blue
color, a trajectory density distribution becomes visible with darker blue at the trajectory mean 
and light blue towards the outer quantiles. It is obvious, that the state at the front of the PFTR
exhibits a more narrow distribition of trajectories than the state at the reactor outlet, because the
residence time of its fluid elements is smaller. Consequently, trajectories differing in parameters have less time to drift apart.
\newline
\newline
The vanilla expected value for the temperature (green) tracks the mean of the system reasonably well
at both positions, since its distance to the darkest part of the density distribition is mostly small.
The lower quantile bound (purple) is located below the mean prediction at all times and positions.
It also portrays a lower boundary for the quantile state as intended. But the accuracy of its quantile coverage,
which is supposed to be equal to $0.1$ cannot be assessed. The same yields for the
upper quantile (yellow), which should cover a fraction of $0.9$ of temperature trajectories. 
However, the quantile models successfully quantify the uncertainty difference between reactor positions
with narrow intervalls at $z=0.25L$ and wider intervals at $z=L$. Smaller intervals seem to be more
challenging to predict, as the $0.9$-quantile NARX excludes the mean prediction at \SI{120}{\second} and
$z=0.25L$. Crucially, the plotted quantile trajectories are simulated not only via recursive quantile temperature predicitions 
but also via the conditioned NARX for the dimensionless concentration measure (fig.~\ref{case:fig:narx framework}).
This may deteriorate the quantile prediction in contrast to the expected value NARX, which relies
completely on least square regression. The according trajectories of $\bm{\chi}$ 
for the concentration measure for ethylene oxide at the same reactor
coordinates are presented (fig.~\ref{results:fig:vanilla_temperature_sim}).

\begin{figure}
    \centering
    \includegraphics[width=1\textwidth]{Figures/Results/test_with_vanilla_2.pdf}
    \caption[Forward simulation of the vanilla NARX framework for the dimensionless concentration.]
    {Forward simulation of the vanilla NARX framework without physics constrained activation for 
    \SI{480}{\second}. The scaled concentration measure of ethylene oxide $\chi_\text{EO}$ is shown
    for two measurement positions $z=0.25L$ and
    $z=L$. $100$ first principle system trajectories are displayed as test data (light blue) for MC sampled parameters.}
    \label{results:fig:vanilla_concentration_sim}
\end{figure}

Here, the expected value NARX (green) $\mathbb{E}$ again follows the trajectory mean sufficiently.
The conditioned quantile trajectories (yellow and purple) however, cannot be accounted for
reliable confidence predictions, as they repeatedly exclude the expected value trajectory.
Especcially, for the position $z=0.25L$ with a tight uncertainty distribition, the 
conditioned models cross the expected value prediction more often. This behavior highlights,
that the conditioned NARX models are not designed for a reliable uncertainty quantification, but represent 
a rather pragmatic approach to allow for independent propagation of temperature quantiles.
Still, they retain some correlational information about the $T$-$\chi$-relationship. 
An increased temperature leads to a faster reaction, thus a higher product and 
low educt concentration and vice versa. This information is conveyed by the quantile-conditioned
NARX models. For ethylene oxide the quantile-conditioned
NARX for the lower temperature quantile, forecasts also lower values at the reactor outlet.
The opposite is achieved by the quantile-conditioned NARX for the upper temperature quantile.
While this correlation is captured to a degree, it is not suited for confidence
interpretation. Furthermore, a distinct probabilistic coverage cannot be guaranteed.
\newline
\newline
To enhance the simulation capability of the quantile-conditioned models, the physics
constrained activation function is employed. The dimensionless concentration
measures act as the conservation quantity at all reactor positions.
When the physics constrained activation is applied naively (fig.~\ref{results:fig:naive_concentration_sim}) (dash-dotted), so only during simulation
but not in training, the trajectories of the quantile-conditioned models do not differ
noticably to the vanilla version (fig.~\ref{results:fig:vanilla_concentration_sim}).
\begin{figure}
    \centering
    \includegraphics[width=1\textwidth]{Figures/Results/test_with_naive_pc_2.pdf}
    \caption[Forward simulation of the naive_pc NARX framework for the dimensionless concentration.]
    {Forward simulation of the naive\_pc NARX framework with physics constrained activation for 
    \SI{480}{\second}. The scaled concentration measure of ethylene oxide $\chi_\text{EO}$ is shown
    for two measurement positions $z=0.25L$ and
    $z=L$. $100$ first principle system trajectories are displayed as test data (light blue) for MC sampled parameters.}
    \label{results:fig:naive_concentration_sim}
\end{figure}

When the fully physics constrained NARX architecture is simulated forward in time, a small improvement
w.r.t the positioning of the quantile-conditioned boundaries can be noticed (fig.~\ref{results:fig:pc-concentration-sim}).
\begin{figure}
    \centering
    \includegraphics[width=1\textwidth]{Figures/Results/test_with_pc_2.pdf}
    \caption[Forward simulation of the pc NARX framework for the dimensionless concentration.]
    {Forward simulation of the pc NARX framework with physics constrained activation for 
    \SI{480}{\second}. The fully physics consistent framework exhibits a minor improvement to the 
    vanilla NARX framework.}
    \label{results:fig:pc-concentration-sim}
\end{figure}
For the majority of time instances, the upper quantile-conditioned NARX (yellow) limits the expected value
trajectory (green) to its top even for the narrow distribition at $z=0.25L$. The same yields for the lower
quantile-conditioned NARX (purple), which limits the expected value to the bottom. Accordingly, the number
of crossing events is reduced. This represent a slightly more meaningful estimation of the concentration measure
by the fully physics constrained NARX framework than by the naive or vanilla versions.
Because the a posteriori application of the physics constrained activation does not truly alter
the resulting trajectories, the physics constrained activation 
is assumed to enhance the learning capability of NARX models.
This statement is underlined by and the significantly faster training convergence.
In summary, model parameters are learned much better for
equation governed data, when according equations are incorporated
as an output activation function into a neural network \cite{chen2024}. The visible effect on
the propageted temperature quantiles is minor, which is why, they are included
within the appendix (figs.~\ref{app:fig:naive-temperature-sim} and \ref{app:fig:pc-temperature-sim}).